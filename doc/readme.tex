\documentclass{article}
\usepackage{amsmath,amssymb}
\usepackage[utf8x]{inputenc}
\usepackage[english]{babel}


\author{I. Sauer, C. Otto, H. Krichene}
\title{Household Resilience Model }
\date{\today}

\begin{document}

\maketitle
\section{ClimateLife}
ClimateLife is the wrapperclass doing the setup of households, hazards and government and manages all the dynamic processes and gathers all the data.

The class handles the following procedure:
\begin{enumerate}
\item Disasters result in losses to households' effective capital stock ($\Delta k_{h}^{eff}$)
\item The diminished asset base generates less income ($\Delta i_{h}$)
\item Reduced income contributes to a decrease in household consumption $\Delta c_{h}$, but affected households must further reduce their consumption to rebuild their damaged assets.
\item Household losses are used to calculate well-being losses ($\delta w_{h}$)
\end{enumerate}

It contains  the weight of each household in the survey and corresponding household weights that account for a representative population
\begin{equation}
  \label{eq:P}
P = \sum_{h=0}^{N_h} w_h.
\end{equation}

\section{Household}
Each household aims at \textbf{minimizing its well-being losses}. This optimization specifies each household's reconstruction and savings expenditure rate, assuming households to \textbf{optimize the fraction of income} they dedicate to repairing and replacing their assets.

\subsection{Household's income}
Income does not need to be calculated, it is given in the FIES!
Households income is equal to the sum of social transfers $i_h^{sp}$ plus the value generated by a household's effective capital stock, less the flat tax at the rate $\delta_{sp}^{tax}$:
\begin{equation}
  \label{eq:ih}
  i_h = i_h^{sp}+(1-\delta_{sp}^{tax}) k_h^{eff} \Pi_k.
\end{equation}

After the shock the dynamic income response to damaged is given as  (the term $\Delta i_h^{PDS}(t)$ is currently omitted):

\begin{equation}
\label{eq:ih}
\Delta i_h(t) = (1-\delta_{sp}^{tax}) \cdot  \Pi_k \cdot \Delta k_h^{eff}(t) 
+  \frac{L(t)}{K} \cdot i_h^{sp}+ \Delta i_h^{PDS}(t).
\end{equation}

\subsection{Household's capital}
The household's capital stock can be derived from income and social transfers reported in the FIES
\begin{equation}
  \label{eq:kh}
k_h^{eff} = \frac{i_h-i_h^{sp}}{\Pi_k(1-\delta_{sp}^{tax})}.
\end{equation}
This formula is given in Step 2 of the tutorial, and $\Pi_{k}$ is the productivity of capital.
The effective capital stock $k_{h}^{eff}$ may be divided into a public $k_{h}^{pub}$ and private capital stock $k_{h}^{prv}$. This is usually done by using the share of private of total asset losses.

\subsubsection{Savings}
We calculate average gap between income and consumption by region and assume that each household maintains one year's surplus

\subsubsection{Reconstruction of capital stock}
Households rebuild their own capital stock, while public assets are rebuild by tax payers. The flood shock effects first on the capital stock and is included taking into account $\Delta k_{h}^{eff}$ derived from the direct damages class. Asset losses are then assumed to be time dependent $\Delta k_{h}^{eff} \rightarrow \Delta k_{h}^{eff}(t)$.\\

Affected households will have to increase their savings rate - that is, avoid consuming some fraction of their post-increase income - to recover these assets. Assuming each household pursues an exponential asset reconstruction pathway, we calculate a reconstruction rate for each household that maximizes its well-being over the X years following the disaster while avoiding bringing consumption below the subsidence line. If this cannot be avoided, it is assumed that it takes place at the pace possible with a saving rate equal to the average saving rate of people living under the subsistence level (according to the FIES). Given these assumptions for the response of each affected household to a disaster (occurring at $t_0$) are given by:
\begin{equation}
\label{eq:kh}
\Delta k_h(t) = \Delta k_h^{eff} \cdot e^{-\lambda_h \cdot t}
\end{equation}
With the construction time $\tau_h$ (years needed until 95\% of lost capital stock are recovered) given by:
\begin{equation}
\label{eq:kh}
\tau_h = ln(\frac{1}{0.05}) \cdot \lambda_h^{-1}
\end{equation}

The reconstruction rate $\lambda_h$ is optimized in order to minimize wellbeing losses. Wellbeing in defined by:

\begin{equation}
\label{eq:kh}
W=\frac{1}{1-\eta} \times  \int [c_h - \Delta c_h(t)]^{1- \eta} \cdot e^{-\rho t}dt
\end{equation}

Expanding these terms and omitting social transfers and taxes for simplicity (savings?) we obtain:

\begin{equation}
\label{eq:kh}
W=\frac{k_h^{eff}}{1-\eta} \times  \int [\Pi - (\Pi+\lambda_h) \cdot ve^{-\lambda_ht}]^{1- \eta} \cdot e^{-\rho t}dt
\end{equation}
We then look for the optimum reconstruction rate $\lambda_h$ to maximize wellbeing over a considered time period $\frac{\partial W}{\partial \lambda} = 0$:
\begin{equation}
\label{eq:kh}
\frac{\partial W}{\partial \lambda} = 0 = \int_{0}^{15} [\Pi - (\Pi+\lambda_h) \cdot ve^{-\lambda_ht}]^{- \eta}(t(\Pi+\lambda_h)-1) \cdot e^{-t(\rho+\lambda)}dt
\end{equation}

\subsection{Household's consumption}
Affected households will also have to forego an additional portion of their income $\Delta c_h^{reco}$ to fund their recovery and reconstruction.Total consumption losses, then, are equal to income losses plus reconstruction costs, less savings and post-disaster support (together represented by $S_h$):
\begin{equation}
\label{eq:kh}
\Delta c_h(t) = \Delta i_h(t) + \Delta c_h^{reco}(t)-S_h
\end{equation}
Total reconstruction costs are equal to the reduction in consumption needed to rebuild their asset stock, plus the increase in taxes needed for the government. The contribution of reconstruction costs to consumption loss at each moment depends on the ownership of the damaged assets, and on the reconstruction rate:
\begin{equation}
\label{eq:kh}
\Delta c_h^(t) = \Delta i_h(t) + \Delta c_h^{reco}(t)-S_h
\end{equation}

\begin{enumerate}
	\item Affected households pay directly and entirely the replacement of the lost assets they owned $\Delta k_{prv}$.
	\item All households pay indirectly and proportionally to their income for the replacement of lost public assets through an extraordinary tax $\Delta k_{pub}$.
	\item Households do not pay for the replacement of the assets they use to generate an income but do not own (e.g. factory where they work $\Delta k_{oth}$	
\end{enumerate}
In order to rebuild at the rate $\lambda_h$ the reconstruction costs to household consumption are given by:
\begin{equation}
\label{eq:kh}
\Delta c_h(t) = \Delta i_h(t) + \Delta c_h^{reco}(t)-S_h
\end{equation}

\begin{equation}
\label{eq:kh}
\Delta c_h^{reco}(t) = \lambda_h \cdot \Delta k_h (t)
\end{equation}

\subsubsection{Optimal consumption of savings and post-disaster support}
Each household uses its savings, plus any post-disaster support it receives to smooth its consumption. Liquid assets are spend to establish a floor, or offset the deepest part of its consumption losses. Assuming an exponential recovery with rate $lambda_h$ and a total value of savings $S_h^{tot}$ the level of this floor $/gamma$ and the time at which the household's savings are exhausted by solving the following coupled equations:
\begin{equation}
\label{eq:kh}
S_h^{tot} + \gamma \hat{t}= \frac{k_h^{eff}v_h}{\lambda_h}(\Pi + \lambda_h)[1-e^{\lambda_h \hat{t}}]
\end{equation}

\begin{equation}
\label{eq:kh}
\gamma = k_h^{eff} [\Pi - v_h(\Pi+\lambda_h)e^{-\lambda_h \hat{t}}]
\end{equation}
The equations are numerically solved:
\begin{equation}
\label{eq:kh}
0 = k_h^{eff} v_h (\Pi+\lambda_h)[1-\beta] + \gamma ln(\beta)-\lambda_h S_h^tot
\end{equation}

where $beta = \gamma \cdot (k_h^{eff} v_h (\Pi+\lambda_h))^{-1}$


\section{Government}
The Government could be something like the administration of e.g. social transfers $i_{h}^{sp}$, it knows the total costs of social transfers
\begin{equation}
  \label{eq:transfer_cost}
\zeta_{sp} = \sum_{h=0}^{N_h} w_ii_h^{sp}.
\end{equation}
There is an additional flat income tax that finances social programs $\operatorname{rate} = \delta_{sp}^{tax}$ that is estimated with the following equation
\begin{equation}
 \label{eq:income_tax}
\sum_{h=0}^{N_{h}} w_{i}i_{h}^{sp} = \sum_{h=0}^{N_h} w_ii_h\frac{\zeta_{sp}}{\sum w_hi_h}.
\end{equation}

Government knows the national income and the national costs of social transfers and distinguish $\delta$,
\begin{equation}
 \label{eq:deltasp}
\delta_{sp}^{tax} = \frac{\zeta_{sp}}{\sum w_hi_h}.
\end{equation}

Additionally, from a modeling perspective we include remittances (spendings from friends and families) in the Government class. Seeing that we cannot account for bilateral flows, we model the remittances as if they would com from one single fund, provided through an additional tax, proportional to income. 

\section{Direct damage}
This is the CLIMADA interface.
Note: We need public and private asset losses to estimate ratio of public and private capital stock, but maybe it is more useful to skip this step and estimate capital shares like it is done in Step 3 of the tutorial, using a fixed ratio.
Households' exposure is interpreted as the probability $f_{a}$ for any given household to be affected.

\begin{equation}
\label{eq:asset losses}
L = \Phi_{a}K = f_{a}\sum_{h=0}^{N_{h}}w_{h}\Delta k_{h}^{eff}
\end{equation}

\begin{equation}
\label{eq:direct loss of hh's capital stock}
\Delta k_{h}^{eff} = v_{h}(k_{h}^{prv}+k_{h}^{pub}+k_{h}^{oth})
\end{equation}

Simplifications: 
\begin{itemize}
	\item disaster only affects only one region at the time
	\item vulnerability is not linked to disaster intensity, a household is either affected or not affected
	\item the vulnerability of a household is given by its private vulnerability
	\item two disasters never happen at the same time, or close enough to have compounding effects.
	
\end{itemize}

\subsection{AIR Worldwide}

As household incomes from the FIES and nominal GDP of a region do not necessarily correspond. The relative asset damages (ratio of total and damaged asset stock)from the AIR model are used. The damage ratio is then applied to the effective capital stock $k_{h}^{eff}$.

Bases on exceedance curves for various return periods and different disaster types.

\subsection{CLIMADA}

Our goal is to couple CLIMADA with the HHRM instead of using the AIR-model. Currently the model takes in the probability for each household within a region to be affected $f_{a}$, while we can generate the flooded fraction at 5km resolution. There are then two options to handle CLIMADA-output:
\begin{itemize}
    \item Given we don't know the exact location of an household, we use a shape-file to cut hazard data and aggregate flooded fractions to gain .
    \item Given we know the exact location of households, we leave the flooded fraction at grid level, assign each households to its closest grid-cell, and choose randomly as many households to be affected as needed to match the fraction, then we would get a $f_{a}$
    \item Note: Here, we can also account for, e.g. unequally affected income-groups.
\end{itemize}
We derive the damage ratio, using the total effective capital stock divided by the damage calculated by CLIMADA. Note: We can also do this grid-cell specific.

In the original version $f_{a}$ ist not estimated from flood data but calibrated. It could be also an option to calibrate $v_{h}$ when we work on grid level.

\end{document}

%%% Local Variables:
%%% mode: latex
%%% TeX-master: t
%%% End:
