\documentclass{article}
\usepackage{amsmath,amssymb}
\usepackage[utf8x]{inputenc}
\usepackage[english]{babel}


\author{I. Sauer, C. Otto}
\title{Mulit-shock resilience model}
\date{\today}

\begin{document}
\maketitle
\section{Agenda for Brian}
\begin{itemize}
	\item Survey design extracting social protection
	\item Recovery of public capital stock
	\item missing income data for Malawi
	\item case study Mozambique
	\item modeling only income from labour
\end{itemize}
\section{Methods}
\subsection{Asset damages}
We model a sequence of damages with CLIMADA flood-module. To match the observed damages as closely as possible, we calibrate the damages to the observed damage recorded in NatCatSERVICE. For each region we estimate the $f_a$ in 
\begin{equation}
L = \Phi_a \cdot K = f_a \cdot \sum\limits_{h=0}^{N_h} w_h k_h^{eff} v_h
\end{equation}
as the fraction of destroyed capital stock in each region. In the original model the following conceptional shift is done:

If exposure is constant for all households in a given area, then we can reinterpret exposure $f_a$
as the fraction of each household affected by a given disaster. After each disaster, of
course, every household will be in exactly one of only two possible states: either it
suffered direct impacts, or it escaped the disaster. On average, however, we can adopt
a probabilistic approach by bifurcating each household in the FIES into two instances:
affected and non-affected. We introduce this split in such a way that the total weight
of each household (as well as asset losses at the household and provincial levels)
remains unchanged:

\begin{equation}
w_h = w_{h_{a}} + w_{h_{na}}=
\begin{cases}
w_{h_{a}} = f_a \cdot w_h \ affected \ households\\
w_{h_{na}} =( 1-f_a) \cdot w_h \ unaffected \ households
\end{cases}
\end{equation}

When we account for multiple shocks and we assume a number of n shocks in one region then we would have to simulate a high number of household states s:
\begin{equation}
s = \binom{n}{k} + \binom{n}{k-1} + \binom{n}{k-2}+ ... + \binom{n}{1}
\end{equation}

with $k = [1,n]  \ k \in \mathbb{N}$.

In the case of the Philippines this would mean  a very large number of combinations ($n \approx 20$) between 1980-2010.\\


\textbf{Possible solutions:}
\begin{itemize}
	\item use a case study with a feasable number of events 
	\item use the flood footprints to decide which events are likely to effect the same households and simulate only likely sequences of events
	
\end{itemize}


\subsection{Recovery}
We extend the original model from Walsh et al., so that we can incorporate a sequence of floods. When a household is hit by a second shock, we assume the same household vulnerability, but apply it on the current capital stock of the household, which might then not be fully recovered.
So for the first shock we assume:

\begin{equation}
\Delta k_h^{eff}(t_{shock}) = v_h \cdot k_h^{eff} 
\end{equation}
And for any following shock we estimate $\Delta k_h^{eff}(t_{shock})$ from the capital stock in recovery:
\begin{equation}
\Delta k_h^{eff}(t_{shock}) = v_h \cdot (k_h^{eff} -  \Delta k_h^{eff}(t_{shock} - 1))
\end{equation}

\end{document}