\documentclass{article}
\usepackage{amsmath,amssymb}
\usepackage[utf8x]{inputenc}
\usepackage[english]{babel}


\author{I. Sauer, C. Otto}
\title{Mulit-shock resilience model}
\date{\today}

\begin{document}
\maketitle
\section{Agenda for Brian}
\begin{itemize}
	\item Survey design extracting social protection
	\item using the income and consumption gap
	\begin{itemize}
	\item In the case of the NCR region, most households can recover rapidly, in 2 to 3 years, and only a few households take more than five years to fully rebuild their asset base. In the
	case of Bicol, with much higher poverty rates, reconstruction is much longer, with a
	significant fraction of households needing more than five years to rebuild their asset
	base. $\rightarrow$ here this only arises due to different vulnerabilities and recovery under the subsidence level.
	\end{itemize}
	\item assumption that one disaster effecs only one region
	\item Recovery of public capital stock
	\item capital owned by others...
	\item missing income data for Malawi
	\item case study Mozambique
	\item modeling only income from labour (over capital from others)
	
\end{itemize}
\section{Methods}
\subsection{Asset damages}
We model a sequence of damages with CLIMADA flood-module. To match the observed damages as closely as possible, we calibrate the damages to the observed damage recorded in NatCatSERVICE. For each region we estimate the $f_a$ in 
\begin{equation}
L = \Phi_a \cdot K = f_a \cdot \sum\limits_{h=0}^{N_h} w_h k_h^{eff} v_h
\end{equation}
as the fraction of destroyed capital stock in each region. In the original model the following conceptional shift is done:

If exposure is constant for all households in a given area, then we can reinterpret exposure $f_a$
as the fraction of each household affected by a given disaster. After each disaster, of
course, every household will be in exactly one of only two possible states: either it
suffered direct impacts, or it escaped the disaster. On average, however, we can adopt
a probabilistic approach by bifurcating each household in the FIES into two instances:
affected and non-affected. We introduce this split in such a way that the total weight
of each household (as well as asset losses at the household and provincial levels)
remains unchanged:

\begin{equation}
w_h = w_{h_{a}} + w_{h_{na}}=
\begin{cases}
w_{h_{a}} = f_a \cdot w_h \ affected \ households\\
w_{h_{na}} =( 1-f_a) \cdot w_h \ unaffected \ households
\end{cases}
\end{equation}

When we account for multiple shocks and we assume a number of n shocks in one region then we would have to simulate a high number of household states s:
\begin{equation}
s = \binom{n}{k} + \binom{n}{k-1} + \binom{n}{k-2}+ ... + \binom{n}{1}
\end{equation}

with $k = [1,n]  \ k \in \mathbb{N}$.

In the case of the Philippines this would mean  a very large number of combinations ($n \approx 20$) between 1980-2010.\\


\textbf{Possible solutions:}
\begin{itemize}
	\item use a case study with a feasable number of events 
	\item use the flood footprints to decide which events are likely to effect the same households and simulate only likely sequences of events
	
\end{itemize}


\subsection{Recovery}
We extend the original model from Walsh et al., so that we can incorporate a sequence of floods. When a household is hit by a second shock, we assume the same household vulnerability, but apply it on the current capital stock of the household, which might then not be fully recovered.
So for the first shock we assume:

\begin{equation}
\Delta k_h^{eff}(t_{shock}) = v_h \cdot k_h^{eff} 
\end{equation}
And for any following shock we estimate $\Delta k_h^{eff}(t_{shock})$ from the capital stock in recovery:
\begin{equation}
\Delta k_h^{eff}(t_{shock}) = v_h \cdot (k_h^{eff} -  \Delta k_h^{eff}(t_{shock} - 1))
\end{equation}
\subsection{Aspects currently missing}
\begin{itemize}
	\item public capital stock \\
	TODO: we distribute $k^{eff}_{h}$ to private $k_h$) and
	public ($k^{pub}_{h}$) using the fraction of private to total asset losses in the region, assuming that (1) the ratios of the different capital categories are similar for all households, and (2) the vulnerability of each household’s public assets is given by the vulnerability of
	its private assets. 
	\begin{itemize}
		\item Does this mean we only optimize over the private capital stock loss but calculate income loss from the public and the private capital stock loss?
	\end{itemize}
	\item we have no timing in our events
	\item subsidence level \\
	TODO:
	If the households cannot avoid having consumption below the subsistence line (for
	instance because consumption is below the subsistence level even without repairing
	and replacing lost assets), then we assume that reconstruction takes place at the pace
	possible with a saving rate equal to the average saving rate of people living at or
	below subsistence level in the Philippines (according to the FIES).
	In addition, households must maintain consumption above a certain level to meet
	their essential needs. To reflect this, we use the following heuristic: if a household
	cannot afford to reconstruct at the optimal rate without falling into subsistence (i.e.
	if ($i_h − \Delta i_h − \lambda_h \  \Delta k^{eff}\ \Biggr|_{t=t_0} < i_{sub}$), then the household reduces its consumption to the subsistence line less the regional savings rate for households in subsistence ($R^{sub}_{sav}$), and uses the balance of its post-disaster income to reconstruct. Its consumption remains at this level until its reconstruction rate reaches the optimum. This leads to an
	initial reconstruction rate equal to $ \lambda_h = \frac{1}{k^{priv}}\cdot  i_h − \Delta i_h − i_{sub} + R_{sub}$
	.
	\item bifurication
	\item early warning system
	\item tax for public recovery
	
\end{itemize}

\end{document}